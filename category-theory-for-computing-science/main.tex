\documentclass[12pt, a4paper]{article}

% Math symbols
\usepackage{amsmath, amsthm, amsfonts, amssymb}
\usepackage{accents}
\usepackage{esvect}
\usepackage{mathrsfs}
\usepackage{mathtools}
\mathtoolsset{showonlyrefs}
\usepackage{cmll}
\usepackage{stmaryrd}
\usepackage{physics}
\usepackage[normalem]{ulem}
\usepackage{ebproof}
\usepackage{extarrows}

% Page layout
\usepackage{geometry}
\usepackage{a4wide}
\usepackage{parskip}
\usepackage{fancyhdr}
\setlength{\headheight}{15pt}

% Font, encoding, russian support
\usepackage[russian]{babel}
\usepackage[sb]{libertine}
\usepackage{xltxtra}

% Listings
\usepackage{listings}
\lstset{basicstyle=\ttfamily,breaklines=true}
\setmonofont{Inconsolata}

% Miscellaneous
\usepackage{array}
\usepackage{calc}
\usepackage{caption}
\usepackage{subcaption}
\captionsetup{justification=centering,margin=2cm}
\usepackage{catchfilebetweentags}
\usepackage{enumitem}
\usepackage{etoolbox}
\usepackage{float}
\usepackage{lastpage}
\usepackage{minted}
\usepackage{svg}
\usepackage{wrapfig}
\usepackage{xcolor}
\usepackage[makeroom]{cancel}

\newcolumntype{L}{>{$}l<{$}}
    \newcolumntype{C}{>{$}c<{$}}
\newcolumntype{R}{>{$}r<{$}}

% Footnotes
\usepackage[hang]{footmisc}
\setlength{\footnotemargin}{2mm}
\makeatletter
\def\blfootnote{\gdef\@thefnmark{}\@footnotetext}
\makeatother

% References
\usepackage{hyperref}
\hypersetup{
    colorlinks,
    linkcolor={blue!80!black},
    citecolor={blue!80!black},
    urlcolor={blue!80!black},
}

\pagestyle{fancy}
\lhead{A course in group theory}
\cfoot{}
\rhead{page \thepage\ of \pageref*{LastPage}}

\theoremstyle{definition}
\newtheorem{exercise}{Exercise}[section]
\newenvironment{solution}[1][Solution.]{\begin{proof}[#1]}{\end{proof}}

\theoremstyle{remark}
\newtheorem*{remark}{Remark}

\newcommand{\R}{\mathbb{R}}
\newcommand{\Q}{\mathbb{Q}}
\newcommand{\Z}{\mathbb{Z}}
\newcommand{\B}{\mathbb{B}}
\newcommand{\N}{\mathbb{N}}
\renewcommand{\Re}{\mathfrak{R}}
\renewcommand{\Im}{\mathfrak{I}}

\newcommand{\const}{\text{const}}
\newcommand{\cond}{\text{cond}}

\DeclareMathOperator*{\xor}{\oplus}
\DeclareMathOperator*{\equ}{\sim}
\DeclareMathOperator{\sign}{\text{sign}}

\DeclarePairedDelimiter{\ceil}{\lceil}{\rceil}

\newcommand{\dbltilde}[1]{\accentset{\approx}{#1}}
\newcommand{\intt}{\int\!}
\renewcommand{\ker}{\text{ker }}
\newcommand{\im}{\text{im }}
\renewcommand{\grad}{\text{grad}}
\newcommand{\rg}{\text{rg}}
\newcommand{\defeq}{\stackrel{\text{def}}{=}}
\newcommand{\defeqfor}[1]{\stackrel{\text{def } #1}{=}}
\newcommand{\itemfix}{\leavevmode\makeatletter\makeatother}
\newcommand{\?}{\textcolor{red}{???}}
\renewcommand{\emptyset}{\varnothing}
\newcommand{\longarrow}[1]{\xRightarrow[#1]{\qquad}}
\DeclareMathOperator*{\esup}{\text{ess sup}}
\newcommand\smallO{
    \mathchoice
    {{\scriptstyle\mathcal{O}}}% \displaystyle
    {{\scriptstyle\mathcal{O}}}% \textstyle
    {{\scriptscriptstyle\mathcal{O}}}% \scriptstyle
    {\scalebox{.6}{$\scriptscriptstyle\mathcal{O}$}}%\scriptscriptstyle
}
\renewcommand{\div}{\text{div}\ }
\newcommand{\rot}{\text{rot}\ }
\newcommand{\cov}{\text{cov}}
\newcommand{\eq}{\stackrel{?}{=}}

\makeatletter
\newcommand{\oplabel}[1]{\refstepcounter{equation}(\theequation\ltx@label{#1})}
\makeatother

\newcommand{\symref}[2]{\stackrel{\oplabel{#1}}{#2}}
\newcommand{\symrefeq}[1]{\symref{#1}{=}}

% xrightrightarrows
\makeatletter
\newcommand*{\relrelbarsep}{.386ex}
\newcommand*{\relrelbar}{%
    \mathrel{%
        \mathpalette\@relrelbar\relrelbarsep
    }%
}
\newcommand*{\@relrelbar}[2]{%
    \raise#2\hbox to 0pt{$\m@th#1\relbar$\hss}%
    \lower#2\hbox{$\m@th#1\relbar$}%
}
\providecommand*{\rightrightarrowsfill@}{%
    \arrowfill@\relrelbar\relrelbar\rightrightarrows
}
\providecommand*{\leftleftarrowsfill@}{%
    \arrowfill@\leftleftarrows\relrelbar\relrelbar
}
\providecommand*{\xrightrightarrows}[2][]{%
    \ext@arrow 0359\rightrightarrowsfill@{#1}{#2}%
}
\providecommand*{\xleftleftarrows}[2][]{%
    \ext@arrow 3095\leftleftarrowsfill@{#1}{#2}%
}

\allowdisplaybreaks

\newcounter{casenum}
\newenvironment{caseof}{\setcounter{casenum}{1}}{\vskip.5\baselineskip}
\newcommand{\case}[2]{\vskip.5\baselineskip\par\noindent {\bfseries Case \arabic{casenum}:} #1\\#2\addtocounter{casenum}{1}}


\begin{document}

\section{Preliminaries}

Skipped due to triviality.

\section{Categories}

\subsection{Basic definitions}

\begin{enumerate}
    \item Prove that sets (as objects) and injective functions (as arrows) form a category with functional composition as the composition operation \(c\).
          \begin{solution}
              Take \(id_A\) to be \(x \mapsto x\), then \(\id_A \circ f = f\) and \(g \circ \id_A = g\) is trivial. The last thing to check is that \(g \circ f\) is injective, that is, whenever \(s \neq s'\), then \(g(f(s)) \neq g(f(s'))\). By injectivity of \(f\), we have \(f(s) \neq f(s')\) and by injectivity of \(g\) we have \(g(f(s)) \neq g(f(s'))\).
          \end{solution}
    \item Do the same as Exercise 1 for sets and surjective functions.
          \begin{solution}
              Let \(f : A \to B, g : B \to C\) be injective functions. Then \(f(A) = B, g(B) = C \Rightarrow g(f(A)) = C\).
          \end{solution}
    \item Show that composition of relations (2.1.14) is associative.
          \begin{solution}
              Let \(\alpha, \beta, \gamma\) be relations from \(A\) to \(B\), from \(B\) to \(C\) and from \(C\) to \(D\).
              \begin{align*}
                  \alpha \circ \beta \circ \gamma
                   & = \{(a, c) \mid \exists b : (a, b) \in \alpha, (b, c) \in \beta\} \circ \gamma          \\
                   & = \{(a, d) \mid \exists b, c : (a, b) \in \alpha, (b, c) \in \beta, (c, d) \in \gamma\} \\
                   & = \alpha \circ (\beta \circ \gamma)
              \end{align*}
          \end{solution}
    \item Prove the following for any arrow \(u : A \to A\) of a category \(\mathcal{C}\). It follows from these facts that C--3 and C--4 of 2.1.3. characterize the identity arrows of a category.
          \begin{enumerate}
              \item If \(g \circ u = g\) for every object \(B\) of \(\mathcal{C}\) and arrow \(g : A \to B\), then \(u = \id_A\).
              \item If \(u \circ h = h\) for every object \(C\) of \(\mathcal{C}\) and arrow \(h : C \to A\), then \(u = \id_A\).
          \end{enumerate}
          \begin{solution}\itemfix
              \begin{enumerate}
                  \item \(\id_A \circ u \defeq u\), but also \(\id_A \circ u = \id_A\) by assumption. \(\Rightarrow u = \id_A\).
                  \item \(u \circ \id_A \defeq u\), but also \(u \circ \id_A = \id_A\) by assumption. \(\Rightarrow u = \id_A\).
              \end{enumerate}
          \end{solution}
\end{enumerate}

\subsection{Functional programming languages}

\begin{enumerate}
    \item \(\texttt{nonzero}: \texttt{NAT} \to \texttt{BOOLEAN}\), subject to equations \(\texttt{nonzero} \circ \texttt{succ} = \texttt{false}\) and \(\texttt{nonzero} \circ \texttt{succ} = \texttt{true}\).
\end{enumerate}

\subsection{Mathematical structures as categories}

\begin{enumerate}
    \item For which sets \(A\) is \(F(A)\) a commutative monoid?
          \begin{solution}
              \(F(A)\) is always a monoid, so the only property to check is commutativity. If \(A = \{\}\), then \(F(A) = \{\}\) and is vacuously commutative. If \(A = \{a\}\), then \(F(A) = \{(), (a), (a, a), \dots\}\) and is commutative. Otherwise, if \(A\) has at least two elements, \(a\) and \(b\), \((a)(b) = (a, b)\), but \((b)(a) = (b, a) \neq (a, b)\), therefore it is not commutative. All in all, \(|A| \leq 1 \Leftrightarrow F(A)\) is commutative.
          \end{solution}
    \item Prove that for each object \(A\) in a category \(\mathcal{C}\), \(\hom(A, A)\) is a monoid with composition of arrows as the operation.
          \begin{solution}
              Take \(\id_A\) as the identity element. Then \(\id_A \circ f = f \circ \id_A = f\) by definition of \(\id\). \(\hom(A, A)\) is closed under composition.
          \end{solution}
    \item Prove that a semigroup has at most one identity element.
          \begin{solution}
              Let \(e_1, e_2\) be identity elements. Then \(e_2 = e_1 e_2 = e_1\), so the identity elements are equal. This is very similar to exercise 2.1.4.
          \end{solution}
\end{enumerate}

\end{document}
