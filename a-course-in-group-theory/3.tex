\begin{exercise}
    Let \(G\) be a group in which \(g^2 = 1\) for all \(g\) in \(G\). Prove that \(G\) is abelian.
\end{exercise}
\begin{solution}
    Proving from the bottom up:
    \begin{align*}
        xy   & = yx            \\
        y    & = x^{ - 1} yx   \\
        yx   & = x^{ - 1} yx^2 \\
        yx   & = x^{ - 1} y    \\
        xy   & = x^{ - 1} y    \\
        xy^2 & = x^{ - 1} y^2  \\
        x    & = x^{ - 1}      \\
        x^2  & = 1
    \end{align*}
    , which holds.
\end{solution}

\begin{exercise}
    Let \(a, b\) and \(c\) be elements of the group \(G\). Find the solutions \(x\) of the equations
    \begin{enumerate}
        \item \(axa^{ - 1} = 1\),
        \item \(axa^{ - 1} = a\),
        \item \(axb = c\) and
        \item \(ba^{ - 1}xab^{ - 1} = ba\)
    \end{enumerate}
\end{exercise}
\begin{solution}\itemfix
    \begin{enumerate}
        \item \begin{align*}
                  axa^{ - 1} & = 1         \\
                  ax         & = a         \\
                  x          & = a^{ - 1}a \\
                  x          & = 1
              \end{align*}
        \item \begin{align*}
                  axa^{ - 1} & = a           \\
                  ax         & = a^2         \\
                  x          & = a^{ - 1}a^2 \\
                  x          & = a
              \end{align*}
        \item \begin{align*}
                  axb & = c                  \\
                  ax  & = cb^{ - 1}          \\
                  x   & = a^{ - 1}c b^{ - 1}
              \end{align*}
        \item \begin{align*}
                  ba^{ - 1}xab^{ - 1} & = ba           \\
                  ba^{ - 1}xa         & = bab          \\
                  ba^{ - 1}x          & = baba^{ - 1}  \\
                  a^{ - 1}x           & = aba^{ - 1}   \\
                  x                   & = a^2ba^{ - 1}
              \end{align*}
    \end{enumerate}
\end{solution}

\begin{exercise}
    Let \(G\) be a group and \(c\) be a fixed element of \(G\). Define a new operation \(*\) on \(G\) by
    \[x * y = xc^{ - 1}y\]
    for all \(x\) and \(y\) in \(G\). Prove that \(G\) is a group under the operation \(*\).
\end{exercise}
\begin{solution}\itemfix
    \begin{enumerate}
        \item Closure is trivial.
        \item \begin{align*}
                  (x * y) * z             & \eq x * (y * z)             \\
                  (xc^{ - 1}y) c^{ - 1} z & \eq xc^{ - 1} (yc^{ - 1} z) \\
              \end{align*}
              , which holds by ``extended associativity'', i.e. that brackets are meaningless.
        \item The neutral element is \(c\):
              \[x * c = x c^{ - 1}c = x 1 = x = 1 x = c c^{ - 1} x = c * x\]
        \item The inverse element is \(c x^{ - 1} c\):
              \[x * c x^{ - 1} c = x c^{ - 1} c x^{ - 1} c = x 1 x^{ - 1} c = x x^{ - 1} c = c\]
              \[c x^{ - 1} c * x = c x^{ - 1} c c^{ - 1} x = c\]
    \end{enumerate}
\end{solution}

\begin{exercise}
    List the orders of all the elements of the group \(D(3)\) of Example 1.9.
\end{exercise}
\begin{solution}\itemfix
    \begin{center}
        \begin{tabular}{l|C|C|C|C|C|C}
            Element & e & a & b & c & d & f \\ \hline
            Order   & 1 & 2 & 2 & 1 & 1 & 1
        \end{tabular}
    \end{center}
\end{solution}

\begin{exercise}
    Give an example of a group \(G\) with elements \(x\) and \(y\) such that \((xy)^{ - 1}\) is not equal to \(x^{ - 1}y^{ - 1}\).
\end{exercise}
\begin{solution}
    \(G = C_4, x = g, y = g^2\)
    \[xy = g^3 \quad (xy)^{ - 1} = g \quad x^{ - 1} = g^3 \quad y^{ - 1} = g^2 \quad x^{ - 1}y^{ - 1} = g^2 \neq (xy)^{ - 1}\]
\end{solution}

\begin{exercise}
    Let \(G\) be a group in which \((xy)^2 = x^2y^2\) for all \(x\) and \(y\) in \(G\). Prove that \(G\) is abelian.
\end{exercise}

\begin{solution}
    \begin{align*}
        xy     & \eq yx   \\
        xxyy   & \eq xyxy \\
        x^2y^2 & = (xy)^2 \\
    \end{align*}
    , which holds by the definition of \(G\).
\end{solution}

\begin{exercise}
    Let \(x\) and \(g\) be elements of a group \(G\). Prove, using mathematical induction, that for all positive integers \(k\),
    \[(x^{ - 1}gx)^k = x^{ - 1}g^kx\]
    Deduce that \(g\) and \(x^{ - 1}gx\) have the same order.
\end{exercise}
\begin{solution}\itemfix
    \begin{itemize}
        \item [Base.] \(k = 0\).
              \[(x^{ - 1}gx)^k = 0 = x^{ - 1}x = x^{ - 1}g^0 x\]
        \item [Induction step.]
              \[(x^{ - 1}gx)^k = x^{ - 1}gx(x^{ - 1}gx)^{k - 1} = x^{ - 1}gx x^{ - 1}g^{k - 1}x = x^{ - 1}g^k x\]
    \end{itemize}

    Order:
    \begin{itemize}
        \item [\( \Rightarrow \)]
              \[g^k = 1 \Rightarrow x^{ - 1}g^k x = 1 \Rightarrow (x^{ - 1}gx)^k = 1\]
        \item [\( \Leftarrow \)]
              \[x^{ - 1}gx = 1 \Rightarrow x^{ - 1}g^kx = 1 \Rightarrow g^kx = x \Rightarrow g^k = 1\]
    \end{itemize}
\end{solution}

\begin{exercise}
    Let \(\omega\) denote the complex number \(e^{2\pi i / 6}\), so that \(\omega^6 = 1\). Let
    \[X = \begin{pmatrix} \omega & 0 \\ 0 & \omega^{ - 1} \end{pmatrix}\]

    Show that \(X^6 = I\) and calculate \(X^{ - 1}\). Find a \(2 \times 2\) matrix \(Y\) such that
    \[XY = YX^{ - 1}\ \mathrm{and}\ Y^2 = X^3.\]

    Show that the set \(G = \{X^i, YX^j : 1 \leq i,j \leq 6\}\) with \(12\) elements is a group under matrix multiplication, and find the order of each element of \(G\).
\end{exercise}

\begin{solution}
    \[X^6 = \begin{pmatrix} \omega^6 & 0 \\ 0 & \omega^{ - 6} \end{pmatrix} = I\]
    \[X^{ - 1} = \begin{pmatrix} \omega^{ - 1} & 0 \\ 0 & \omega \end{pmatrix}\]
    Let \(Y = \begin{pmatrix} a & b \\ c & d \end{pmatrix}\).

    \[Y^2 = \begin{pmatrix} a^2 + bc & ab + bd \\ ac + cd & bc + d^2  \end{pmatrix} = X^3 = \begin{pmatrix} \omega^3 & 0 \\ 0 & \omega^{ - 3} \end{pmatrix}\]
    \[XY = \begin{pmatrix} a\omega & b\omega \\ c \omega^{ - 1} & d \omega^{ - 1} \end{pmatrix} \quad YX^{ - 1} = \begin{pmatrix} a \omega^{ - 1} & b \omega \\ c \omega^{ - 1} & d \omega \end{pmatrix}\]
    This implies that \(a = d = 0\). Therefore \(bc = \omega^3 = - 1\). Let \(c = - b^{ - 1}\).

    The following is a proof of \(G\) being a group.
    \begin{enumerate}
        \item \(\{X^i : 1 \leq i \leq 6\}\) is isomorphic to \(C_6\) by a map that takes the first element of the first row and an inverse map \(\omega^i \mapsto \begin{pmatrix} \omega^i & 0 \\ 0 & \omega^{ - i} \end{pmatrix}\). Closure of \(\{X^i\}\) is therefore trivial. Moreover, \(YX^j \times X^i = YX^{j + i \mod 6} \in G\). The following is the proof of two other cases.
              \begin{align*}
                  X^i \times YX^j & =
                  \begin{pmatrix}
                      \omega^i & 0             \\
                      0        & \omega^{ - i}
                  \end{pmatrix} \times \left(\begin{pmatrix}
                          0         & b \\
                          -b^{ - 1} & 0
                      \end{pmatrix} \times \begin{pmatrix}
                          \omega^j & 0             \\
                          0        & \omega^{ - j}
                      \end{pmatrix}\right) \\
                                  & =
                  \begin{pmatrix}
                      \omega^i & 0             \\
                      0        & \omega^{ - i}
                  \end{pmatrix} \times \begin{pmatrix}
                      0                   & b\omega^{ - j} \\
                      - b^{ - 1} \omega^j & 0
                  \end{pmatrix}                                                \\
                                  & =
                  \begin{pmatrix}
                      0                         & b\omega^{i - j} \\
                      - b^{ - 1} \omega^{j - i} & 0
                  \end{pmatrix}                                                                                  \\
                                  & = YX^{j - i \mod 6}
              \end{align*}
              \begin{align*}
                  YX^i \times YX^j & =
                  \begin{pmatrix}
                      0                   & b\omega^{ - i} \\
                      - b^{ - 1} \omega^i & 0
                  \end{pmatrix} \times \begin{pmatrix}
                      0                   & b\omega^{ - j} \\
                      - b^{ - 1} \omega^j & 0
                  \end{pmatrix} \\
                                   & = \begin{pmatrix}
                      \omega^{j - i} & 0              \\
                      0              & \omega^{i - j}
                  \end{pmatrix}              \\
                                   & = X^{j - i \mod 6}
              \end{align*}
        \item Matrix product is associative.
        \item The identity matrix is the identity element and is \(X^6\).
        \item The inverse for \(X^i\) is \(X^{6 - i}\), for \(YX^i\) is \(YX^{6 - i}\), which follows from the closure proof.
    \end{enumerate}
\end{solution}
