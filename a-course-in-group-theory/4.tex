\begin{exercise}
    Which of the following sets \(H\) are subgroups of the given group \(G\)?
    \begin{enumerate}
        \item \(G\) is the set of integers under addition, \(H\) is the set of even integers;
        \item \(G = S(3), H = \{1, (12), (23), (13)\}\);
        \item \(G = GL(2, \R), H\) is the set of matrices of the form \(\begin{pmatrix} 1 & a \\ 0 & 1 \end{pmatrix}\), where \(a\) is any real number.
    \end{enumerate}
\end{exercise}

\begin{solution}\itemfix
    \begin{enumerate}
        \item The identity element of \(\Z_{+}\) is \(0\), which is contained in \(H\). Moreover, even integers are closed under addition and the additive inverse of an integer is even. Therefore, all conditions of 4.2 (2) hold.
        \item No, \((12)(23) \notin G\).
        \item Let \(A_a\) denote \(\begin{pmatrix} 1 & a \\ 0 & 1 \end{pmatrix}\). \(A_a \times A_b = A_{a + b} \in H\), because reals are closed under addition. The identity element of \(G\), \(I\) is \(A_0\) and is therefore contained in \(H\). The inverse of \(A_a\) is \(A_{ - a}\), which follows from the statements above.
    \end{enumerate}
\end{solution}

\begin{exercise}
    Give an example of a group \(G\) with subgroups \(H\) and \(K\) such that \(H \cup K\) is not a subgroup of \(G\).
\end{exercise}
\begin{solution}
    Let \(G\) be an abelian group with distinct elements \(1, k, h, hk\). Let \(H = \ev{h} = \{1, h\}, K = \ev{k} = \{1, k\}\). \(H \cup K\) does not contain \(hk\), but contains \(h\) and \(k\).
\end{solution}

\begin{exercise}
    Let \(G\) be the group in Question 2 of Exercises 1. Find the number of elements in \(\ev{A, D}\). Is \(\ev{A, C}\) cyclic? Write down the multiplication table for \(\ev{B, F}\).
\end{exercise}
\begin{solution}
    \(\ev{A, D} = \{I, A, D, B, E, G, C, F\}, |\ev{A, D}| = 8\)

    \(\ev{A, C} = \{I, A, C, B\}\). This group is cyclic, which can be seen from its' Cayley table (see \ref{ex12})

    \(\ev{B, F} = \{I, B, F, D\}\)
    \begin{center}
        \begin{tabular}{C|C|C|C|C}
              & I & B & F & D \\ \hline
            I & I & B & F & D \\
            B & B & I & D & F \\
            F & F & D & I & B \\
            D & D & F & B & I \\
        \end{tabular}
    \end{center}
\end{solution}

\begin{exercise}
    Let \(G\) be the group with presentation
    \(\{x,y : x^4 = 1, x^2 = y^2, xy = yx^{ - 1}\}\).
    Decide how many elements are in \(G\) and determine its multiplication table.
\end{exercise}
\begin{solution}
    Consider the order of \(y\). Since \(y^4 = x^4 = 1\), it is \(\leq 4\).

    If \(y^3 = 1\), \(x^2y = 1\) and therefore \(1 = xyx^{ - 1}\), which implies \(y = 1, x^2 = 1\), which contradicts the definition of \(G\).

    If \(y^2 = 1, x^2 = 1\), which contradicts the definition of \(G\). From here onward, I will use the symbol ``\textreferencemark'' as a shorthand.

    This proves \(y^4 = 1\).

    As per the argument given in the chapter, \(xy^i = yx^i\) for all \(i\).

    Clearly, \(G\) contains all 3 powers of \(x\) and \(y\). Let's consider \(xy\).

    \begin{caseof}
        \case{\(xy = 1\)}{
            \(y = x^{ - 1} = x^3, 1 = xy = yx^{ - 1} = x^3 x^{ - 1} = x^2\), \textreferencemark
        }
        \case{\(xy = x\)}{
            \(y = 1, x = xy = yx^{ - 1} = x^{ - 1} \Rightarrow x^2 = 1\), \textreferencemark
        }
        \case{\(xy = x^2\)}{
            \(y = x, x^2 = xy = yx^{ - 1} = yx^3 = x^4 = 1\), \textreferencemark
        }
        \case{\(xy = x^3\)}{
            \(y = x^2 = y^2 \Rightarrow y = 1\), see case 2. % `\ref` doesn't work in `caseof'
        }
        \case{\(xy = y\)}{
            \(x = 1\), \textreferencemark
        }
        \case{\(xy = y^3\)}{
            \(x = y^2 = x^2 \Rightarrow x = 1\), \textreferencemark
        }
    \end{caseof}

    This proves that \(xy\) is in fact a distinct element of \(G\). Let's consider \(xy^3\) now.

    \begin{caseof}
        \case{\(xy^3 = 1\)}{
            \(1 = xy^3 = yx \Rightarrow y = x^{ - 1} = x^3, 1 = xy^3 = x x^9 = x^2\), \textreferencemark
        }
        \case{\(xy^3 = x\)}{
            \(y^3 = 1 \Rightarrow x^2y = 1 \Rightarrow 1 = xyx^{ - 1} \Rightarrow y = 1 \Rightarrow x^2 = 1\), \textreferencemark
        }
        \case{\(xy^3 = x^2\)}{
            \(x^3y = x^2 \Rightarrow xy = x\), see case 2 for \(xy\).
        }
        \case{\(xy = x^3\)}{
            \(y = x^2, y^2 = x^2 \Rightarrow y = 1\), see case 2 for \(xy\)
        }
        \case{\(xy = y\)}{
            \(x = 1\), \textreferencemark
        }
        \case{\(xy = y^3\)}{
            \(x = y^2, x^2 = y^2 \Rightarrow x = 1\), \textreferencemark
        }
    \end{caseof}

    This proves that \(xy^3\) is a distinct element of \(G\). The following Cayley table proves closure:

    \begin{center}
        \begin{tabular}{C|C|C|C|C|C|C|C|C}
                 & 1    & x    & x^2  & x^3  & y    & y^3  & xy   & xy^3 \\ \hline
            1    & 1    & x    & x^2  & x^3  & y    & y^3  & xy   & xy^3 \\
            x    & x    & x^2  & x^3  & 1    & xy   & xy^3 & y^3  & y    \\
            x^2  & x^2  & x^3  & 1    & x    & y^3  & y    & xy^3 & xy   \\
            x^3  & x^3  & 1    & x    & x^2  & xy^3 & xy   & y    & y^3  \\
            y    & y    & xy^3 & xy   & y    & x^2  & 1    & x^3  & x    \\
            y^3  & y^3  & xy   & y    & xy^3 & 1    & x^2  & x    & x^3  \\
            xy   & xy   & y    & xy^3 & y^3  & x^3  & x    & x^2  & 1    \\
            xy^3 & xy^3 & y^3  & xy   & y    & x    & x^3  & 1    & x^2  \\
        \end{tabular}
    \end{center}
\end{solution}
