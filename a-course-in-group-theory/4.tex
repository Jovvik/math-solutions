\begin{exercise}
    Which of the following sets \(H\) are subgroups of the given group \(G\)?
    \begin{enumerate}
        \item \(G\) is the set of integers under addition, \(H\) is the set of even integers;
        \item \(G = S(3), H = \{1, (12), (23), (13)\}\);
        \item \(G = GL(2, \R), H\) is the set of matrices of the form \(\begin{pmatrix} 1 & a \\ 0 & 1 \end{pmatrix}\), where \(a\) is any real number.
    \end{enumerate}
\end{exercise}

\begin{solution}\itemfix
    \begin{enumerate}
        \item The identity element of \(\Z_{+}\) is \(0\), which is contained in \(H\). Moreover, even integers are closed under addition and the additive inverse of an integer is even. Therefore, all conditions of 4.2 (2) hold.
        \item \? Didn't understand the notation. If this is cycle notation, then \(1 = (12)\), which doesn't make sense.
        \item Let \(A_a\) denote \(\begin{pmatrix} 1 & a \\ 0 & 1 \end{pmatrix}\). \(A_a \times A_b = A_{a + b} \in H\), because reals are closed under addition. The identity element of \(G\), \(I\) is \(A_0\) and is therefore contained in \(H\). The inverse of \(A_a\) is \(A_{ - a}\), which follows from the statements above.
    \end{enumerate}
\end{solution}

\begin{exercise}
    Give an example of a group \(G\) with subgroups \(H\) and \(K\) such that \(H \cup K\) is not a subgroup of \(G\).
\end{exercise}
\begin{solution}
    Let \(G\) be an abelian group with distinct elements \(1, k, h, hk\). Let \(H = \ev{h} = \{1, h\}, K = \ev{k} = \{1, k\}\). \(H \cup K\) does not contain \(hk\), but contains \(h\) and \(k\).
\end{solution}

\begin{exercise}
    \(\ev{A, D} = \{I, A, D, B, E, G, C, F\}, |\ev{A, D}| = 8\)

    \(\ev{A, C} = \{I, A, C, B\}\). This group is cyclic, which can be seen from its' Cayley table (see \ref{ex12})

    \(\ev{B, F} = \{I, B, F, D\}\)
    \begin{center}
        \begin{tabular}{C|C|C|C|C}
              & I & B & F & D \\ \hline
            I & I & B & F & D \\
            B & B & I & D & F \\
            F & F & D & I & B \\
            D & D & F & B & I \\
        \end{tabular}
    \end{center}
\end{exercise}

\begin{solution}
    Consider the order of \(y\). Since \(y^4 = x^4 = 1\), it is \(\leq 4\).

    If \(y^3 = 1\), \(x^2y = 1\) and therefore \(1 = xyx^{ - 1}\), which implies \(y = 1\) and \(G = C_3, C_2\) or \(C_1\).

    If \(y^2 = 1, x^2 = 1\) and therefore \(G\) is the only group with \(3\) elements, \(C_3\) \textit{(or \(C_1\))}.
\end{solution}